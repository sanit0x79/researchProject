\documentclass[conference]{IEEEtran}

\usepackage{graphicx} % For figures
\usepackage{amsmath}  % For mathematical symbols
\usepackage{lipsum}   % For filler text

\begin{document}

% Title
\title{Evaluating Trade-offs in Throughput, Latency, and Power Consumption for ESP32 Access Point Configurations}

% Authors
\author{\IEEEauthorblockN{Evan Dokter}
\IEEEauthorblockA{
Studentnummer: 439545}
}

% Begin document
\maketitle

% Abstract
\begin{abstract}
This research investigates the trade-offs in throughput, latency, and power consumption for an ESP32 configured as an access point in different Wi-Fi modes (802.11b, 802.11g, and 802.11n). By analyzing experimental results, the study provides insights into the optimal configurations for varying application requirements, emphasizing practical applications in IoT systems.
\end{abstract}

% Keywords
\begin{IEEEkeywords}
ESP32, Wi-Fi, Access Point, Throughput, Latency, Power Consumption, IoT
\end{IEEEkeywords}

\section{Introductie}
The ESP32 is a popular microcontroller used in IoT applications due to its low cost, versatility, and built-in Wi-Fi capabilities. This paper evaluates its performance as an access point, focusing on key metrics such as throughput, latency, and power consumption when configured in various Wi-Fi modes. \lipsum[1-2]

\section{Achtergrond}
Wi-Fi modes (802.11b, g, and n) differ in terms of maximum throughput, range, and power efficiency. \lipsum[3-4]

\section{Methodes}
\subsection{Hardware Setup}
The experiment uses an ESP32 board, a USB power meter, and a laptop running `iperf` for throughput and latency testing. \lipsum[5]

\subsection{Wi-Fi Configuratie}
The ESP32 is programmed using Arduino IDE, and its Wi-Fi protocol is set using the \texttt{WiFi.softAPsetProtocol()} function. The power consumption is measured during idle and data transfer states. \lipsum[6]

\section{Resultaten en Analyse}
\subsection{Throughput}
The throughput for each mode was measured using `iperf`. Figure~\ref{fig:throughput} shows the results. \lipsum[7]

\begin{figure}[htbp]
    \centering
    \includegraphics[width=\columnwidth]{example.png} % Replace with your figure
    \caption{Throughput comparison across Wi-Fi modes.}
    \label{fig:throughput}
\end{figure}

\subsection{Latency}
Ping tests reveal varying round-trip times (RTTs) for each mode. \lipsum[8]

\subsection{Power Consumption}
Figure~\ref{fig:power} depicts power usage for each Wi-Fi mode. \lipsum[9]

\begin{figure}[htbp]
    \centering
    \includegraphics[width=\columnwidth]{example2.png} % Replace with your figure
    \caption{Power consumption comparison across Wi-Fi modes.}
    \label{fig:power}
\end{figure}

\section{Discussie}
The trade-offs in throughput, latency, and power consumption highlight that the optimal mode depends on application requirements. \lipsum[10-12]

\section{Conclusie}
This study provides a detailed analysis of the ESP32's performance as an access point in different Wi-Fi modes. Future work may explore dynamic mode switching to optimize performance based on network conditions. \lipsum[13]

\begin{thebibliography}{1}
\bibitem{ref1} A. Author, ``Title of the paper,'' \emph{Journal Name}, vol. X, no. Y, pp. 1--10, Year.

\bibitem{ref2} B. Author, \emph{Title of the Book}. Publisher, Year.
\end{thebibliography}

\end{document}
